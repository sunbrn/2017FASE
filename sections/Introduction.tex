Several multi-valued model checking techniques, such as~\cite{bruns1999model,bruns2000model,gurfinkel2003multi,bruns2004MCmultivalued}, have been proposed to support the verification of models that are \emph{incomplete}.
A model is incomplete whenever its state space is not fully specified.
%i.e., it stands for a functionality that still needs to be refined.
Three-valued model checking extends classical two-valued model checking by returning an additional \emph{maybe} value whenever the property satisfaction depends on how the incompleteness is removed.
More precisely, a three-valued model checking algorithm returns one of the following values: \emph{true} if the property $\phi$ \emph{definitely holds} (it is satisfied regardless of how incomplete parts are refined), \emph{false} if $\phi$ \emph{does not hold} (a violating behavior which does not depend on the incomplete parts exists), and \emph{maybe} if $\phi$ \emph{possibly holds} (its satisfaction depends on the parts to be refined).

In the classical context of two-valued model checking, algorithms do not provide the designer with enough information.
Although a sample violating behavior is normally returned when the property is violated, no equally useful insight is provided if the property is proved to hold. 
Indeed, in this case, it might be useful to receive a formal explanation of the reason \emph{why} the system satisfies the property.
To achieve this goal, the verification framework can be equipped with a deductive verification engine that formally justifies why the model checking procedure has failed in the search of a counterexample by exploiting the state space generated during the search.
Deductive verification algorithms have been developed for fully specified models~\cite{peled2001model,peled2001falsification}, but no known similar approach deals with incomplete models.
This is a remarkable limitation in practical contexts based on incremental development, where the designer may start by providing an initial, high-level version of the model, which is iteratively narrowed down as design progresses and uncertainties are removed.
Whenever the verification answer is true or maybe, the proof can be used by the designer as a guidance in the refinement process, and as a confirmation on the correctness of the design choices already performed. 
In some cases, it may even suggest that actually the property does not capture the intended correctness condition, and it should be modified.
For this reason the integration of deductive verification techniques in a multi-valued context can guide the designer towards a correct development.

This paper proposes \NAME , a THRee valued Integrated Verification Engine for incomplete models.
\NAME\ enriches model checking for incomplete models with a deductive verification engine that justifies whether the verified system definitely satisfies or possibly satisfies the property of interest.
Whenever the property \emph{definitely holds}, the deductive verification approach proves that neither a definitely violating nor a possibly violating behavior can be found in the current instance of the model.
If the property \emph{possibly holds}, the model checker returns a possible counterexample, which describes a possible behavior that violates the property. 
In addition, the deductive verification engine returns to the designer a proof that specifies why a violating behavior cannot be found in the current model.
Finally, whenever the property \emph{does not hold}, a counterexample which specifies a violating behavior is returned.

We consider the model $M$ and the property $\phi$ expressed respectively as a Partial Kripke Structure (PKS)~\cite{bruns1999model} and a Linear Temporal Logic (LTL)~\cite{pnueli1977temporal} formula, but the proposed framework can be also extended to support other formalisms.
The framework is founded on previous theoretical results~\cite{bruns2000model,godefroid2005MCvsGMC,godefroid2011ltl,peled2001model}, which have been modified to fit into the proposed approach. It exploits the formal definition of the three-valued semantics of LTL formulae over infinite words presented in~\cite{godefroid2011ltl} and uses a model checking algorithm---in line with the one presented in~\cite{bruns2000model,gurfinkel2003multi}.
The model checking algorithm is equipped with a customization of the deductive verification framework proposed in~\cite{peled2001model}.
We discuss the applicability of the proposed integrated verification environment considering the three-valued~\cite{bruns1999model} and the thorough~\cite{bruns2000model} LTL semantics.
Furthermore, we also analyze the validity of \NAME\ over a particular subset of LTL formulae called self-minimizing~\cite{godefroid2005MCvsGMC} LTL formulae.

%DELETED
%It exploits the formal definition of the three-valued semantics of LTL formulae over infinite words presented in~\cite{godefroid2011ltl}, which is adapted for application to PKSs. 
%Based on this semantics, we formally describe a model checking algorithm---in line with the one presented in~\cite{bruns2000model,gurfinkel2003multi}---.
%Finally, the model checking algorithm is equipped with a customization of the deductive verification framework proposed in~\cite{peled2001model}.


The benefits of the approach are evaluated using a simple semaphore example.
The model of the semaphore is obtained by abstracting the one presented in~\cite{katoen2008} and is considered w.r.t. three LTL properties.
\NAME\ helps the designer in understanding which choices of component design lead to (in)correct behavior of the system, and why some choices imply that a property definitely/possibly holds.
This information provides the designer with good insights on how to refine/modify the model of the system.

\begin{sloppypar}
The rest of the paper is organized as follows. 
Section~\ref{sec:motivating} introduces the semaphore example.
Section~\ref{sec:preliminaries} presents the modeling formalisms, their properties and verification procedures upon which this work is based.
Section~\ref{sec:contribution} describes \NAME.
Section~\ref{sec:preliminaryEvaluation} evaluates the approach on the semaphore example.
Section~\ref{sec:discussion} discusses the applicability of the approach.
Section~\ref{sec:stateoftheart}  presents the related works.
Finally, Section~\ref{sec:conclusions} concludes the paper.
\end{sloppypar}
