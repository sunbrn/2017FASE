Three-valued~\cite{larsen1988modal,godefroid2001abstraction,bruns1999model,bruns2000model,benevs2009checking,godefroid2011ltl} and multi-valued~\cite{gurfinkel2003multi,bruns2004MCmultivalued} model checking techniques have been used to verify models that are incomplete, i.e., in which some information is missing. 
Two types of semantics are usually considered in these works: compositional (three-valued) and/or non-compositional (thorough).
The compositional semantics exploits the Kleene algebraic structure between the values $\{false, \bot, true \}$. 
The non-compositional semantics is  based on the completeness preorder.
These two semantics have been analyzed in many works, such as~\cite{gurfinkel2005thorough,godefroid2005MCvsGMC}.
Depending on the modeling formalism, different model checking techniques have been developed. For example, several works proposed in literature consider Partial Kripke Structures (e.g.,~\cite{bruns1999model,bruns2000model,godefroid2011ltl,gurfinkel2003multi,bruns2004MCmultivalued}), others Modal Transition Systems (e.g.,~\cite{larsen1988modal,godefroid2001abstraction}). 
However, to the best of our knowledge, none of these approaches has been combined with deductive verification.  

Deductive verification includes a set of techniques that establish the validity of the formula (for more information see~\cite{manna2012temporal}).
Some works have used deductive verification as a means to enrich the information returned by model checking~\cite{gurfinkel2003proof,clarke1995efficient}.
In other techniques~\cite{namjoshi2001certifying,cleaveland2002evidence,rajan1995integration}, the idea is to exploit the structure of the state space generated by the model checker to explain why a property holds.
To the best of our knowledge, none of these approaches has been applied in a multi-valued context.


