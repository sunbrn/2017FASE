We have proposed \NAME, a framework that enhances a three-valued model checker with a deductive verification engine. 
Whenever the property of interest is definitely satisfied, or possibly satisfied, \NAME\ provides the designer with additional information regarding why a certain result is returned by the model checker.
The proof gives intuition on what is working correctly in the current design and insights for the next development rounds.
The  presented framework can be implemented on top of existing model checking tools based on automata theory.

\NAME\ has been evaluated considering a simple semaphore example, which showed  the usefulness  of the approach.
We proved that the result generated by \NAME\ is always valid when self-minimizing LTL formulae are considered.
We discussed that self-minimizing formulae can be constructed following specific patterns and that checking whether an LTL formula is self-minimizing can be done using appropriate procedures.
We also argued that most of the properties of practical interest are already self-minimizing and that some LTL formulae can be transformed into a self-minimizing version.
We showed that when non self-minimizing LTL formulae are considered, \NAME\ produces a valid result in most of the cases: when a $\LTLtrue$ or $\LTLfalse$ value is returned, and also when a $?$ value is returned and the formula is possibly satisfied under the thorough semantics.
Finally, we pointed out that \NAME\ improves as model checking and deductive verification frameworks enhance. 

As future work, we aim to implement \NAME\ by integrating  existing model checking and theorem proving frameworks.
Additionally, we will prove the usefulness of \NAME\ over a real world example.