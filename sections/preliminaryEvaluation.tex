This section discusses how \NAME\ supports the designer in the development of the system that satisfies the properties of interest.
We first consider the example introduced in Section~\ref{sec:motivating} w.r.t. property $\phi_2$. Then, properties $\phi_1$ and $\phi_3$ are also briefly discussed.


\textbf{Property $\boldsymbol{\phi_2}$.} 
The property $\phi_2=\LTLglobally \LTLfinally green$ specifies that green lights up infinitely often.
Even if in state $s_1$ the green light is on, the property is ``only" possibly satisfied.
\NAME\ returns the possible counterexample $(s_0, s_2)^\omega$, that shows why $\phi_2$ is possibly satisfied.
Specifically,  by looping an infinite number of time on states $s_0$ and $s_2$ the green light will not be turned on.
This is a possible counterexample since the value of the $green$ proposition in $s_2$ is currently unspecified.
Since the property $\phi_2$ is possibly satisfied, the search of a definitive counterexample has failed.
\NAME\ computes a proof that explains the motivation.
To search for a counterexample, the intersection automaton $M_{opt}$ presented in Figure~\ref{fig:productOpt} is explored by \NAME\ during the proof computation. The obtained proof is presented in Table~\ref{table:proof}, where $g$ stands for $green$.
The nodes that have been analyzed in different steps are circled  in Figure~\ref{fig:productOpt} through different grey frames.
(\emph{Step1}). \NAME\ analyzes the failed nodes, i.e., the nodes in the set $Fail(I_{opt})$.
Since in these nodes the search for a counterexample trivially fails, the formula associated with the corresponding states of $\mathcal{A}_{\lnot\phi_2}$ holds.
Thus, since the state $\langle s_1, q_1 \rangle$ of the intersection automaton is a failed node, the formula $green \LTLor \LTLnext \LTLfinally green$ (valid in $q_1$) is satisfied by the model state $s_1$.
This formula is effectively true in $s_1$ since the green light is on.
By means of a similar reasoning, the algorithm states that the property  $green \LTLor \LTLnext \LTLfinally green$ also holds in $s_2$. Indeed, since in the optimistic model the green light is on in $s_2$, the search of a counterexample has failed in $\langle s_2, q_1 \rangle$.
(\emph{Step2}). Since all the successors of $\langle s_0, q_1 \rangle$ satisfy $green \LTLor \LTLnext \LTLfinally green$, it is possible to deduce that this property is also satisfied in $s_0$.
(\emph{Step3}). The induction rule is applied considering the strongly connected component  $\left \{ \langle s_0, q_0 \rangle, \langle s_1, q_0 \rangle, \langle s_2, q_0 \rangle \right \}$.
The rule allows to conclude that $s_0$ satisfies the property $\LTLnext \LTLglobally \LTLfinally green$.
(\emph{Step4}). \NAME\ applies the conjunction rule to $s_0$.
Since $s_0$ satisfies both $\LTLnext \LTLglobally \LTLfinally green$ and $green \LTLor \LTLnext \LTLfinally green$, it is possibly to deduce that $s_0$ satisfies the property $\phi_2$.
This provides an interesting insight to the designer: if she/he does her/his ``best" to satisfy the property (she/he turns the green light on in $s_2$) the property becomes satisfied. The proof clearly states why.


\begin{table}[t]
\centering
\caption{Proof that $\phi_2$ is not violated.}
\label{table:proof}
\begin{tabular}[b]{ | p{1.8cm} | p{2.1cm} | p{4.3cm} | p{3cm} |  }
\hline
\emph{Step 1} &
\emph{Step 2} &
\emph{Step 3} &
\emph{Step 4}\\
\hline
%%\multicolumn{4}{|c|}{Steps}  \\
%% \hline
\textbf{Fail} & 
\textbf{Successors} &
\textbf{Induction} &
\textbf{Conjunction}\\
 \hline
 $\langle s_2,q_1\rangle,$\newline
$ \langle s_1,q_1\rangle$
 &
 $\langle s_0,q_1\rangle$ 
 &
 $\mathcal{X}=\{\langle s_0,q_0\rangle,\langle s_1,q_0\rangle,\langle s_2,q_0\rangle\}$
 $Exit(\mathcal{X})=\{\langle s_0,q_1\rangle,\langle s_1,q_1\rangle,$ $\langle s_2,q_1\rangle\}$
 &
The initial node $s_0$ \\	
 \hline
 \inferrule{s_1 \in Fail(I_{opt})\\
 s_2 \in Fail(I_{opt})
}{ s_1\models g \LTLor \LTLnext \LTLfinally g \\
 s_2\models g \LTLor \LTLnext \LTLfinally g}
  & 
\inferrule{s_0\rightarrow\{s_1,s_2\} \\
 s_1\models g \LTLor \LTLnext \LTLfinally g \\
  s_2\models g \LTLor \LTLnext \LTLfinally g }
 {s_0\models g \LTLor \LTLnext \LTLfinally g} 
  &
 \inferrule{ 
 s_0\models g \LTLor \LTLnext \LTLfinally g \\
  s_1\models g \LTLor \LTLnext \LTLfinally  g \\
  s_2\models g \LTLor \LTLnext \LTLfinally g \\
s_0\rightarrow \{s_1,s_2 \} \\\\
 s_1\rightarrow \{s_0 \} \\\\
 s_2\rightarrow \{s_0 \}   
   }
 {s_0\models \LTLnext \LTLglobally \LTLfinally g \\
 s_1\models \LTLnext \LTLglobally \LTLfinally g \\
 s_2\models \LTLnext \LTLglobally \LTLfinally g }
  &
\inferrule{
 s_0\models \LTLnext \LTLglobally \LTLfinally g \\
  s_0\models g \LTLor \LTLnext \LTLfinally g \\
 \LTLnext \LTLglobally \LTLfinally g \LTLand  (g \LTLor  \LTLnext \LTLfinally g)\rightarrow \phi_2 }
 {s_0\models \phi_2}  \\
\hline

\end{tabular}
\end{table}



\textbf{Property $\boldsymbol{\phi_1}$.}
The property $\phi_1=\LTLglobally \LTLfinally red$ specifies that red lights up infinitely often.
The designer wants to know whether the model of Figure~\ref{fig:modelmot} satisfies $\phi_1$.
Intuitively, it is sufficient to observe that the proposition $red$ is evaluated with $\LTLtrue$ in the state $s_0$ and that the system always returns to this state (after visiting either $s_1$ or $s_2$).
\NAME\ returns $\LTLtrue$ as its verification result and produces a proof that highlights how and why a definite counterexample is not found in the graph. First, it identifies the nodes $\langle s_0, q_1 \rangle$ and $\langle s_2, q_1 \rangle$ as failed. The conclusions found on these nodes are propagated to the node $\langle s_1, q_1 \rangle$. All the successors of the SCC formed by the intersection states related to the property state $q_0$ are analyzed. Therefore conclusions are drawn also on this SCC. 
The proof is not reported for brevity and it follows the same pattern as the one in Table~\ref{table:proof}.


\textbf{Property $\boldsymbol{\phi_3}$.}
$\LTLglobally (red \LTLimplication \LTLglobally green)$ cannot be satisfied by the chosen model. Therefore \NAME\ returns a definite counterexample, e.g. $(s_0, s_1)^\omega$.

