We present Partial Kripke Structures~\cite{bruns1999model} (PKSs), which is the formalism we assume the designer has used to describe the system under development.
We discuss the three-valued~\cite{bruns1999model} and thorough~\cite{bruns2000model} semantics of LTL formulae over PKSs, since the knowledge of these two semantics is necessary to understand when \NAME\ can be applied.
Finally, we introduce the verification procedure of an LTL formula over a PKS for the three-valued semantics, which is the base block upon which our deductive verification procedure is constructed.


\emph{Partial Kripke Structures} (PKSs) extend KSs by allowing a proposition in a given state to be labelled with $?$ to represent an unknown value of a proposition. 

\begin{definition}[Partial Kripke Structure \cite{bruns1999model} (PKS)]
A PKS $M$ is a tuple $\langle S, R, S_0,$ $ AP, L \rangle$,
where:
\begin{enumerate*}
\item[] $S$ is a set of \emph{states};
\item[] $R\subseteq S\times S$ is a \emph{left-total} \emph{transition relation} on $S$;
\item[] $S_0$ is a set of initial states;
\item[] $AP$ is a set of atomic propositions;
\item[] $L: S\times AP\rightarrow \{\top,?,\bot\}$ is a \emph{function} that, for each state in $S$, associates a truth value in the set $\{\top,?,\bot\}$ to every atomic proposition in $AP$.
\end{enumerate*}  
\end{definition}

The grade crossing semaphore, shown in Figure~\ref{fig:modelmot} and described in Section~\ref{sec:motivating}, is an example of how the system can be represented by means of a PKS.

A \emph{completion} of a PKS $M$ is a KS $M^\prime$ that completes $M$ by assigning values to the unknown propositions.
We will use $\mathcal{C}(M)$ to indicate all the completions of a PKS $M$.


Two different semantics of LTL can be considered: the three-valued or the thorough semantics.

%----------------------------------------------------------------------------------------------------------------------------------------------------------------
% Three-valued LTL semantic
%----------------------------------------------------------------------------------------------------------------------------------------------------------------
The \emph{three-valued LTL semantics} specifies that a formula $\phi$ definitely holds in a PKS $M$ if it is true for all possible values of the unknown propositions in $M$. 
Likewise, it is definitely violated if it is false despite the unknown values.
%A formula $\phi$ is not satisfied if there is a path $\pi$ in the PKS which violates $\phi$ despite the unknown values of the atomic propositions associated to the states of $\pi$.
%Otherwise the formula is unknown.

\begin{definition}[Three-valued LTL semantics~\cite{godefroid2011ltl}]
Given a PKS $M = \langle S, R,$ $S_0, AP, L \rangle$, a path $\pi=s_0,s_1,\ldots$, and a formula $\phi$, the three-valued semantics $[(M,\pi)\models\phi]$ is defined inductively as follows:
\begin{align*}
&[(M,\pi) \models p] & \buildrel \text{def}\over = &&& L(s_0,p)\\
&[(M,\pi) \models \lnot\phi] & \buildrel \text{def}\over = &&& \textnormal{comp}([(M,\pi) \models\phi]) \\
&[(M,\pi) \models \phi_1 \LTLand \phi_2] & \buildrel \text{def}\over =   &&& \min([(M,\pi) \models\phi_1],[(M,\pi) \models\phi_2])\\
&[(M,\pi) \models \LTLnext \phi] & \buildrel \text{def}\over =  &&& [(M,\pi^1) \models\phi]\\
&[(M,\pi) \models \phi_1 \LTLuntil \phi_2] & \buildrel \text{def}\over =  &&& \max_{j\geq 0}(\min(\{[(M,\pi^i) \models\phi_1]|i<j\} \cup \{[(M,\pi^j) \models\phi_2]\}))
\end{align*}
\end{definition}


The conjunction (disjunction) is defined as the minimum (maximum) of its arguments, following the order $\bot<?<\top$. Negation is defined by the function comp (complement), which maps $\top$ to $\bot$, $\bot$ to $\top$, and $?$ to $?$. These functions are extended to sets considering min($\emptyset$)=$\top$ and max($\emptyset$)=$\bot$.

Given a PKS $M = \langle S, R,$ $S_0, AP, L \rangle$, a state $s$, and a formula $\phi$,  $[(M, s) \models \phi] \buildrel \text{def}\over =   \min(\{[(M, \pi) \models \phi] \mid \pi^0=s\})$.
Intuitively this means that, given a formula $\phi$, each state $s$ of $M$ is associated with the minimum of the values obtained considering the LTL semantics over any path $\pi$ that starts in $s$.



%&[(M,\pi) \models \phi_1 \LTLuntil \phi_2] & = &&& max(\{min(\{[(M,\pi^i) \models\phi_1]|i<j\} \cup \{[(M,\pi^j) \models\phi_2]\})|j\geq 0\})\\


A PKS $M$ \emph{definitely satisfies} a property $\phi$ ($[M \models \phi]=\top$) iff for all initial states $s_0 \in S_0$ of $M$, $[(M, s_0) \models \phi]=\top$. 
A PKS $M$ \emph{does not satisfy} the property $\phi$ ($[M \models \phi]=\bot$) iff there exists an initial state $s_0 \in S_0$ of $M$ such that  $[(M, s_0) \models \phi]=\bot$.
A PKS \emph{possibly satisfies} $\phi$ otherwise.
In the rest of the paper we will use interchangeably the expressions ``$M$ definitely satisfies $\phi$'' and ``$\phi$ holds in $M$'', ``$M$ does not satisfy $\phi$'' and ``$\phi$ does not hold in $M$'', as well as ``$M$ possibly satisfies $\phi$'' and ``$\phi$ possibly holds in $M$''.



The three-valued semantics does not behave always as expected: there are cases in which $\phi$ possibly holds but all the KSs completions (obtained by replacing the $?$ values with $\top$ and $\bot$) actually satisfy (or do not satisfy)  $\phi$.
For instance, this happens when $\phi$ is a tautology or it is unsatisfiable with a traditional two-valued interpretation. 

The \emph{thorough LTL semantics}---differently from the three-valued semantics---specifies that a formula is possibly satisfied only if there exist two completion KSs $M_1$ and $M_2$, that definitely satisfy and violate $\phi$, respectively.

%DELETED
%Let the \emph{completions} $\mathcal{C}(M)$ of a PKS $M$ be the set of all the complete KS. Every,  KS $M_i \in \mathcal{C}(M)$ is such that, for each state, to every atomic proposition with value $?$ either the value $\top$ or $\bot$ is assigned.

\begin{definition}[Thorough semantics~\cite{bruns2000model}]
\label{def:thoroughsemantics}
Let $\phi$ be an LTL formula and $M$ a PKS, the truth value of a formula $\phi$ of a thorough temporal logic semantics, written $[M\models\phi]_t$, is defined as follows:

\begin{equation}
 [M \models\phi]_t \buildrel \text{def}\over = 
                						\begin{cases}
                  								\top & \quad  \text{if}\ M^\prime \models \phi \text{ for all } M^\prime \in \mathcal{C}(M)\\
                  								\bot & \quad  \text{if}\ M^\prime \not\models\phi \text{ for all } M^\prime \in \mathcal{C}(M)\\
                  								? &  \quad  \text{otherwise}
                							  \end{cases}      					  		   
\end{equation}

\end{definition}

\begin{proposition}[\cite{godefroid2011ltl}] 
\label{def:mcImpliesGmc}
Given a PKS and an LTL formula $\phi$,
\begin{enumerate}
\item $[M\models\phi] =$ $ \top \Rightarrow$ $ [M\models\phi]_t = \top$.
\item $[M\models\phi] = \bot \Rightarrow [M\models\phi]_t = \bot$.
\end{enumerate}
\end{proposition}
%
That is, a formula which is true (false) under the three-valued semantics is also true (false) under the thorough semantics.

Given the three-valued and the thorough semantics, there exists a subset of LTL formulae, indicated in literature as \emph{self-minimizing}, such that the two semantics coincide. Formally,

\begin{proposition}[\cite{godefroid2005MCvsGMC}]
\label{def:self-minimizing}
 Given a model $M$ and a  self-minimizing LTL property $\phi$, then $[M\models\phi]=[M\models\phi]_t$.
\end{proposition}
%In~\cite{godefroid2005MCvsGMC} the authors showed that many temporal-logic formulae of practical interest are self-minimizing.
%More precisely, all the formulae that do not contain atomic propositions with mixed polarity (negated and not negated) in their negation normal form are self-%minimizing; if the formulae follow specific patterns are self-minimizing;
%some LTL formulae that are not self-minimizing can be converted into self-minimizing LTL formulae using a procedure called semantic-minimization.
%The interested reader may refer to~\cite{godefroid2005MCvsGMC} for more details.
%
%

In the following, we present a model checking algorithm for LTL formulae when a three-valued semantics is considered. 
The algorithm exploits two classical model checking procedures for KSs.
These procedures consider a version of $M$, called \emph{complement-closed} structure~\cite{bruns2000model}, in which for every proposition $p \in AP$, there exists a new proposition $q$, called complement-closed proposition, such that $L(s,p)=$ comp$(L(s,q))$, for all $s \in S$.
The proposition $q$, whose value is complementary to the one of $p$, is indicated as $\overline{p}$.
For example, the complement-closed version of the PKS of the semaphore example is presented in Figure~\ref{fig:model}, where the atomic propositions $\overline{g}$ and $\overline{r}$ are associated with the complement of the values of  propositions green ($g$) and red ($r$), respectively.

\begin{figure}[t]
\begin{minipage}[b]{.5\textwidth}
  \begin{figure}[H]
 \centering
\input{images/modelApproachNew.tex}
\caption{The PKS of  the  crossing semaphore.}
\label{fig:model}
\end{figure}
\end{minipage}
\begin{minipage}[b]{.5\textwidth}
  \begin{figure}[H]
\centering
\input{images/property_2.tex}
\caption{The BA associated with $\phi_2$.}
\label{fig:property}
\end{figure}
\end{minipage}
\end{figure}



%
%\begin{definition}[PKS optimistic and pessimistic labelling~\cite{bruns2000model}]
%Given a three-valued labelling function $L$,  we define the derived optimistic ($L_{opt}$) and pessimistic ($L_{pes}$) labelling function for every state s of the model $M$ as follows: 
%\begin{equation}
%    L_{opt}(s,p) \buildrel \text{def}\over = 
%                					\begin{cases}
%                  						\top & \quad  \text{if}\ L(s,p)=?\\
%                  						L(s,p) &  \quad otherwise
%                					\end{cases}
%\end{equation}            					  
%\begin{equation}      					  
%   L_{pes}(s,p) \buildrel \text{def}\over = 
%	                            \begin{cases}
%                  				       \bot  & \quad \text{if}\ L(s,p)=?\\
%                  						L(s,p) & \quad otherwise
%                					\end{cases}
%\end{equation}
%\end{definition}

The model checking procedure for a PKS $M$ is based on an optimistic and pessimistic approximation of $M$'s complement-closure.
The optimistic (pessimistic) approximation function $L_{opt}$  ($L_{pes}$) associates the value $\top$ ($\bot$) to each atomic proposition of the complement-closure of $M$ with value $?$.
Thus, in the optimistic (pessimistic) approximation, the propositions $g$ and $r$ associated with the value $?$ are assigned the value $\top$ ($\bot$).
Given a PKS $M=\langle S,R, S_0, L \rangle$, we have $M_{pes}=\langle S,R, S_0, L_{pes} \rangle$ for the pessimistic structure, and $M_{opt}=\langle S,R, S_0, L_{opt} \rangle$ for the optimistic one.
%
%The model checking procedure exploits the under and the over approximations as we show hereafter.

The three-valued model checking algorithm assumes that property $\phi$ is rewritten using complement-closed propositions. 
The rewriting procedure works in two steps.
First, the formula is expressed such that negations only appear in front of atomic propositions. 
Second, each negated proposition is substituted by the corresponding complemented proposition.


\begin{theorem}[Three-valued model checking~\cite{bruns2000model}\footnote{Note that Theorem~\ref{th:threevaluedMC} is stated for Positive Propositional Modal Logic in~\cite{bruns2000model} but is proved to be also valid for LTL (see \cite{bruns2000model,godefroid2005MCvsGMC,godefroid2011ltl}).}]
\label{th:threevaluedMC}
Let $\phi$ be an LTL formula obtained using the procedure just discussed, $M=\langle S, R, S_0, L\rangle $ a PKS with $s \in S$, and $M_{pes}$ and $M_{opt}$ the corresponding pessimistic and optimistic structures. Then
\begin{equation}
    [(M,s)\models\phi]  = 
                						 \begin{cases}
                  				 				\top & \quad \text{if}\ (M_{pes},s)\models\phi\\
                  								\bot & \quad \text{if}\ (M_{opt},s)\not\models\phi\\
                  								? & \quad  otherwise
                						\end{cases}
\end{equation}
\end{theorem}

This technique exploits two runs of the classical two-valued model checking performed respectively on a pessimistic and an optimistic completion of the incomplete model $M$.
Intuitively, in the construction of the optimistic completion $M_{opt}$, the algorithm ``tries to do its best" to build a KS which satisfies $\phi$. 
If a violating behavior is found in $M_{opt}$, then a definitive counterexample is returned since the property $\phi$ does not hold.
Vice versa, in the construction of the pessimistic completion $M_{pes}$, the three-valued model checker tries to construct a KS which violates $\phi$. 
If no violating behavior is found in $M_{pes}$, then $\phi$  definitely holds.
Otherwise, it could be the case where there exists some completion in which $\phi$ holds and others in which it does not hold. 
This implies that a $?$ value is returned by the three-valued model checker.
Note that whenever a $\top$ or $\bot$ value is returned, the property $\phi$ is guaranteed to be true/false in all the completions of $M$.
For additional details on this procedure see~\cite{bruns2000model,godefroid2011ltl}.




%\begin{theorem}(The verification procedure is correct)
%\end{theorem}
%\begin{proof}
%Given a self-minimizing temporal logic formula $\phi$, by Proposition~\ref{def:self-minimizing}, generalized (which is based on the thorough semantic) and
%three-valued model checking produce the same result.
%Thus, the considered three-valued model checking algorithm returns the correct value also under the thorough interpretation.
%\end{proof}


Properties $\phi_1$, $\phi_2$ and $\phi_3$ of the crossing semaphore example are respectively satisfied, possibly satisfied and not satisfied for the model $M$ of Figure~\ref{fig:model}.

Let us consider property $\phi_2$. The procedure first checks if $\phi_2$ is violated.
The intersection $I_{opt}$  between the optimistic structure $M_{opt}$ (obtained by changing all the $?$ values in $\LTLtrue$) and the BA $\mathcal{A}_{\neg\phi_2}$ associated with the property $\neg \phi_2$ (represented in Figure~\ref{fig:property}) is computed.
The automaton $I_{opt}$ is presented in Figure~\ref{fig:productOpt}.
The transitions marked with dashed-lines and the grey frames do not belong to the intersection: their meaning will be described in the following.
The state space of  $I_{opt}$ is explored in search of a definitive counterexample. 
Since in $I_{opt}$ no state labelled with an accepting state of  $\mathcal{A}_{\lnot\phi_2}$ (i.e., $\{q_1\}$) can be entered infinitely often, no counterexample is identified.
Then, the procedure checks if $\phi_2$ is possibly violated.
The intersection $I_{pes}$ between the pessimistic structure $M_{pes}$ (obtained by changing all the $?$ values in $\LTLfalse$) and $\mathcal{A}_{\neg\phi_2}$ is computed.
The automaton $I_{pes}$ is presented in Figure~\ref{fig:productPess}.
The state space of  $I_{pes}$ is explored in search of a possibly violating behavior.
Since it exists a state labelled with an accepting state of  $\mathcal{A}_{\lnot\phi_2}$ (i.e., $\{q_1\}$) which can be entered infinitely often (both $\langle s_2, q_1\rangle$ and $\langle s_0, q_1\rangle$), a possible counterexample is returned. 
Consequently, $\phi_2$ is possibly satisfied, i.e., there exists a completion $M_{pes}$ of $M$ that violates $\phi_2$.
For example, if $g=\bot$ and  $r=\bot$ to the propositions in $s_2$, we obtain a violating behavior when the system moves alternatively between $s_0$ and $s_2$. 




\begin{figure}[t]
\begin{minipage}[b]{.5\textwidth}
  \begin{figure}[H]
\centering
\input{images/productOpt_2.tex}
\caption{Product $I_{opt}=M_{opt}\otimes\mathcal{A}_{\lnot\phi_2}$}
\label{fig:productOpt}
 \end{figure}
\end{minipage}%
\begin{minipage}[b]{.5\textwidth}
  \begin{figure}[H]
\centering
\input{images/productPes_2.tex}
\caption{Product $I_{pes}=M_{pes}\otimes\mathcal{A}_{\lnot\phi_2}$}
\label{fig:productPess}
 \end{figure}
\end{minipage}%
\end{figure}